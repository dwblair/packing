\documentclass[]{article}
\usepackage[T1]{fontenc}
\usepackage{lmodern}
\usepackage{amssymb,amsmath}
\usepackage{ifxetex,ifluatex}
\usepackage{fixltx2e} % provides \textsubscript
% use microtype if available
\IfFileExists{microtype.sty}{\usepackage{microtype}}{}
\ifnum 0\ifxetex 1\fi\ifluatex 1\fi=0 % if pdftex
  \usepackage[utf8]{inputenc}
\else % if luatex or xelatex
  \usepackage{fontspec}
  \ifxetex
    \usepackage{xltxtra,xunicode}
  \fi
  \defaultfontfeatures{Mapping=tex-text,Scale=MatchLowercase}
  \newcommand{\euro}{€}
\fi
\ifxetex
  \usepackage[setpagesize=false, % page size defined by xetex
              unicode=false, % unicode breaks when used with xetex
              xetex]{hyperref}
\else
  \usepackage[unicode=true]{hyperref}
\fi
\hypersetup{breaklinks=true,
            bookmarks=true,
            pdfauthor={},
            pdftitle={},
            colorlinks=true,
            urlcolor=blue,
            linkcolor=magenta,
            pdfborder={0 0 0}}
\setlength{\parindent}{0pt}
\setlength{\parskip}{6pt plus 2pt minus 1pt}
\setlength{\emergencystretch}{3em}  % prevent overfull lines
\setcounter{secnumdepth}{0}

\author{}
\date{}

\begin{document}

\subsection{About}

Statistical physics-y approaches to finding optimal packings (highest
density arrangements of shapes within shapes)

Initially simple python code; the hope is to extend this with cython.

\subsection{Algorithm}

\subsubsection{Verison 0.1 (current)}

\begin{itemize}
\item
  Two types of moves: a) translation (of an individual circle); b)
  expansion / contraction (of all circles). At each step, the algorithm
  chooses one of these moves (type ``a'' being much more likely); in the
  case of move type a), a random particle is chosen for an attempted
  random translation -- if it overlaps with another particle, then the
  move is rejected. In the case of move type b), an attempt is made to
  expand or contract all particles (biased towards expansion); if no
  overlap, move is accepted (which is always the case if contraction)
\end{itemize}

\subsubsection{Version 0.2:}

\begin{itemize}
\item
  Nearest neighbor list, or cell, overlap checking
\item
  Contraction should be accepted with probability P = exp(-Beta dV) --
  dV is change in volume of system
\item
  Expansion and translation amounts should be periodically adjusted
  depending on relative size of particles in system -- do a
  `thermalization' of these paramters, then do `equilibration' \ldots{}
  repeat \ldots{}
\item
  Parallel tempering -- do more than one system at a time
\item
  Population annealing -- implement this algorithm
\end{itemize}

\end{document}
